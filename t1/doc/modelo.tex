\documentclass[
	12pt,
	oneside,
	a4paper,
	english,
	french,
	spanish,
	brazil,
	]{abntex2}

\usepackage{lmodern}
\usepackage[T1]{fontenc}
\usepackage[utf8]{inputenc}
\usepackage{indentfirst}
\usepackage{color}
\usepackage{graphicx}
\usepackage{microtype}

\usepackage[brazilian,hyperpageref]{backref}
\usepackage[alf]{abntex2cite}

\renewcommand{\backrefpagesname}{Citado na(s) página(s):~}
\renewcommand{\backref}{}
\renewcommand*{\backrefalt}[4]{
	\ifcase #1
		Nenhuma citação no texto.
	\or
		Citado na página #2.
	\else
		Citado #1 vezes nas páginas #2.
	\fi}

\titulo{Primeiro Trabalho Prático}
\autor{Tulio Brunoro de Souza}
\local{Vitória ES}
\data{2020}

\makeatletter
\hypersetup{
     	%pagebackref=true,
		pdftitle={\@title}, 
		pdfauthor={\@author},
    	pdfsubject={\imprimirpreambulo},
	    pdfcreator={LaTeX with abnTeX2},
		pdfkeywords={abnt}{latex}{abntex}{abntex2}{relatório técnico}, 
		colorlinks=true,
    	linkcolor=blue,
    	citecolor=blue,
    	filecolor=magenta,
		urlcolor=blue,
		bookmarksdepth=4
}
\makeatother

\setlength{\parindent}{1.3cm}
\setlength{\parskip}{0.2cm}

\begin{document}

\selectlanguage{brazil}
\frenchspacing

\imprimircapa

\textual

\chapter{Introdução}
descrição do problema a ser resolvido e visão geral sobre o funcionamento do
programa (em termos de módulos - TADs, arquivos, etc.).

O problema consiste em implementar um sistema de construção colaborativa de uma
enciclopédia, provendo uma interface para que o usuário possa realizar suas
tarefas.

O sistema da enciclopédia consiste de páginas, editores, colaborações e links.
Todos esses componentes juntos compõem o sistema do WikiED. Uma página consiste
de colaborações, feitas por editores, e links.

Lista encadeada de páginas;
Lista encadeada de colaborações;
Contribuições consistem de texto, fornecido por arquivo, e editor;
Inserir elementos ao final da lista;

\chapter{Metodologia}
descrição da implementação do programa. Devem ser detalhadas as estruturas de
dados utilizadas (preferencialmente com diagramas ilustrativos), o
funcionamento das principais funções utilizadas, bem como decisões tomadas
relativas aos casos e detalhes de especificação que porventura estejam omissos
no enunciado. Modularize o seu programa usando a técnica de tipos abstratos de
dados, como discutido em sala de aula.

\chapter{Conclusão}
comentários gerais sobre o trabalho e as principais dificuldades encontradas em
sua implementação.

\phantompart

\bibliography{main}

\end{document}
